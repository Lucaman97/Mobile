\documentclass[a4paper, 50pt, twoside]{article}
\usepackage[italian]{babel}
\usepackage[a4paper, top=2cm, bottom=2cm, left=3cm, right=3cm]{geometry}
\usepackage{graphicx}
\graphicspath{{immagini/}}
\usepackage{chngcntr}
\counterwithin{figure}{section}
\usepackage{braket}
\usepackage{amsmath}
\usepackage{fancyhdr}
\usepackage{xcolor}
\pagestyle{fancy}
\lfoot{EasyVersity}

\begin{document}


\title{Relazione progetto Programmazione mobile}
\date{Settembre, 2019}
\author{Tomassini Danilo, Simone Cappella, Luca Mannini \\ Ingegneria Informatica e dell'Automazione}
\maketitle
\vspace*{\fill}
\vspace*{\fill}

\newpage
\tableofcontents{}

\newpage
\section{Obbiettivi}
EasyVersity rappresenta uno strumento di supporto per lo studente.

Permette di gestire:
\begin{itemize}
\item \textbf{Orario:} salva l'orario delle lezioni nella tua applicazione per consultarlo quando vuoi.
\item \textbf{Archivio appunti locale:} da la possibilità di salvare appunti raggruppandoli per materia, indicando titolo e data si può contestualizzare al meglio l'appunto in questione.
\item \textbf{Condivisione appunti:} rende possibile la condivisione ed il download degli appunti.
\item \textbf{Impostazioni:} da qui si possono cambiare informazioni come username e password, eprendere visione di info "about us".
\end{itemize}



\end{document}